\section{Cycles and Permutations}
\begin{definition}
    Given a list $a_1,a_2,\ldots,a_k$ of distinct elements of $\{1,2,\ldots,n\}$, the $k$-cycle, written as $(a_1\ a_2\ \ldots\ a_k)$ is the permutation $\pi\in S_n$ defined by $\pi(a_i)=a_{i+1}$ for $1\le i\le k-1$, $\pi(a_k)=a_1$, and $\pi(j)=j$ if $j\notin\{a_i:1\le i\le k\}$.
\end{definition}
It is obvious that it is indeed a permutation, and we know that
$$(a_1\ a_2\ \ldots\ a_k)^{-1}=(a_k\ a_{k-1}\ \ldots\ a_1)$$
In particular, there is a set of special cycles.
\begin{definition}
    A $2$-cycle is called a transposition.
\end{definition}
Two different cycles $(a_1\ a_2\ \ldots\ a_k),(b_1\ b_2\ \ldots\ b_l)$ are called disjoint if $\{a_i:1\le i\le k\}\cap\{b_j:1\le j\le l\}=\varnothing$.\\
We can compose the permutations since they are in fact functions
\begin{example}
    $((1\ 2\ 3\ 4)\circ(3\ 2\ 4))(1)=2$, and $((1\ 2\ 3\ 4)\circ(3\ 2\ 4))(2)=(1\ 2\ 3\ 4)(4)=1$, similarly $((1\ 2\ 3\ 4)\circ(3\ 2\ 4))(3)=3,((1\ 2\ 3\ 4)\circ(3\ 2\ 4))(4)=4$, so it is a transposition $(1\ 2)$.\\
    By the same way, we have $(3\ 2\ 4)\circ (1\ 2\ 3\ 4)=(1\ 4)$, so $(3\ 2\ 4)$ and $(1\ 2\ 3\ 4)$ do not commute.\\
    So $S_4$ is not abelian.
    In fact, $S_n$ is not abelian for all $n\ge 3$ since $S_n$ can be embedded into $S_{n+1}$ by fixing the last element.
\end{example}
\begin{lemma}
    1. $(a_1\ a_2\ \ldots\ a_k)=(a_k\ a_1\ a_2\ \ldots\ a_{k-1})$.\\
    2. Disjoint cycles commute.
\end{lemma}
\begin{proof}
    Trivial.
\end{proof}
\begin{theorem}\label{disjoint_cycles}
    Every permutation is a product of disjoint cycles (including $1$-cycles for convenience) and it is unique to write it as such a product up to the ambiguity stated in the preceding lemma.
\end{theorem}
\begin{proof}
    Trivial but let us write this down.
    We do strong induction on $n$.
    The base case is trivial.
    Now we consider the sequence $1, \sigma(1),\sigma^2(1),\ldots$.
    Since there is only finitely many numbers, at some point the sequence goes back to $1$.
    Indeed, there must be some $p>q$ such that $\sigma^p(1)=\sigma^q(1)$, thus $\sigma^{p-q}(1)=1$.
    So we take the smallest $k\ge 1$ such that $\sigma^k(1)=1$, so $\sigma^i(1)\neq\sigma^j(1)$ for $i\neq j$ because if so then $\sigma^{i-j}(1)=1$ (WLOG $i>j$) which contradicts the minimality of $k$.\\
    Hence $\sigma$ maps $S=\{1,\sigma(1),\sigma^2(1),\ldots\}$ to itself, so it is a permutation on $T\{1,2,\ldots,n\}\setminus S$, thus $\sigma|_T$ is a product of disjoint cycles by induction hypothesis, thus we must have $\sigma=(1\ \sigma(1)\ \sigma^2(1)\ \ldots\ \sigma^{k-1}(1))\sigma|_T$, which proves the existence.\\
    The uniqueness is obvious since if $\sigma$ is written as two cycles, then we can pick an element and by enumerating it we can show that the cycles containing that element are the same.
    Hence the theorem is proved.
\end{proof}
\begin{lemma}
    For any permutation $\sigma$, write it as the product of disjoint cycles, that is,
    $$\sigma=(a_1^1\ a_2^1\ \ldots\ a_{k_1}^1)(a_1^2\ a_2^2\ \ldots\ a_{k_2}^2)\cdots (a_1^r\ a_2^r\ \ldots\ a_{k_r}^r)$$
    Then the order of $\sigma$ is the LCM of $k_1,k_2,\ldots,k_r$.
\end{lemma}
\begin{proof}
    Note that
    $$\sigma^j=(a_1^1\ a_2^1\ \ldots\ a_{k_1}^1)^j(a_1^2\ a_2^2\ \ldots\ a_{k_2}^2)^j\cdots (a_1^r\ a_2^r\ \ldots\ a_{k_r}^r)^j$$
    since disjoint cycles commute.
    Now the order of a cycle $(a_1^i\ a_2^i\ \ldots\ a_{k_i}^i)$ has order $k_i$.
    So if we let $\ell$ be the LCM of $k_1,k_2,\ldots,k_r$, we immediately have $\sigma^\ell=e$.
    Suppose $\sigma^m=e$ for some $m$, then we would have
    $$(a_1^1\ a_2^1\ \ldots\ a_{k_1}^1)^m=((a_1^2\ a_2^2\ \ldots\ a_{k_2}^2)^m\cdots (a_1^r\ a_2^r\ \ldots\ a_{k_r}^r)^m)^{-1}$$
    But the LHS fixes $a_s^1$ for any $1\le s\le k_1$ due to disjointness, thus the LHS must equal to the identity, i.e. $k_1|m$.
    Similarly $k_i|m$ for any $1\le i\le r$, thus $\ell|m\implies \ell\le m$, hence $\ell$ is the order of $\sigma$.
\end{proof}
\subsection{The Sign of Permutation}
\begin{proposition}
    Every permutation is a product of transpositions.
\end{proposition}
\begin{proof}
    By Theorem \ref{disjoint_cycles}, it suffices to show that every cycle is a product of transpositions, but
    $$(a_1\ a_2\ \ldots\ a_k)=(a_1\ a_k)(a_1\ a_{k-1})\cdots (a_1\ a_2)$$
    As desired.
\end{proof}
\begin{proof}[Alternative proof]
    We proceed by induction on $n$ such that the permutation is in $S_n$.
    In $n=0$, there is nothing to show.
    More generally, for $\sigma\in S_n$ for $n>1$.
    Choose an $a\in\{1,2,\ldots,n\}$, then $(a\ \sigma(a))\sigma$ fixes $a$, hence is a permutation on at most $n-1$ elements, so the proof is done by induction.
\end{proof}
\begin{definition}
    Define the sign of a permutation $\sigma$ by
    $$
    \operatorname{sgn}(\sigma)=
    \begin{cases}
        1\text{, if $\sigma$ has an even number of transpositions}\\
        -1\text{, otherwise}
    \end{cases}
    $$
\end{definition}
\begin{proposition}
    The sign of the permutation is well-defined.
\end{proposition}
\begin{proof}
    Let $\sigma$ be written as the product of disjoint cycles (including $1$-cycles).
    Suppose there is $\ell(\sigma)$ many such cycles.
    It is well-defined by Theorem \ref{disjoint_cycles}.\\
    Consider a transposition $(c\ d)$.
    $\ell(\sigma\circ (c\ d))=\ell(\sigma)+1$ if $c,d$ are in the same cycle in $\sigma$.
    Otherwise, it is $\ell(\sigma)-1$.
    So in either case, $\ell(\sigma\circ (c\ d))\equiv \ell(\sigma)+1\pmod{2}$.\\
    Note then that $\ell(\sigma)=\ell(t_1t_2\cdots t_k)$ where $t_i$ are transpositions.
    Thus $\ell(\sigma)\equiv\ell(e)+k\equiv n+k\pmod{2}$, so if a permutation has both signs $k,l$, then $k\equiv l\pmod{2}$.
    Therefore the sign is well-defined.
    Indeed, $\operatorname{sgn}(\sigma)=(-1)^{\ell(\sigma)-n}$.
\end{proof}
Immediately $\operatorname{sgn}((a_1\ a_2\ \ldots\ a_r))=(-1)^{r-1}$.
\begin{corollary}
    The function $\operatorname{sgn}$ is a homomorphism $S_n\to (\{1,-1\},\times,1)$.
\end{corollary}
\begin{proof}
    Trivial.
\end{proof}
\begin{definition}
    A permutation $\sigma$ is even if $\operatorname{sgn}(\sigma)=1$ and it is odd otherwise.
\end{definition}
\begin{definition}
    The alternating group $A_n\le S_n$ is defined as $A_n=\ker\operatorname{sgn}$.
    That is, $A_n$ consists of all even permutations.
\end{definition}
Note that any transposition has sign $-1$ and the identity has sign $1$, thus $\operatorname{sgn}$ is surjective, therefore the index of $A_n$ is $2$, hence it is normal.
\subsection{Conjugation in the Permutation Group}
\begin{proposition}
    If we have a permutation $\sigma$, then $\sigma (a_1\ a_2\ \ldots\ a_r)\sigma^{-1}=(\sigma(a_1)\ \sigma(a_2)\ \ldots\ \sigma(a_r))$.
\end{proposition}
\begin{proof}
    Trivial.
\end{proof}
\begin{corollary}
    $\tau,\tau'\in S_n$ are conjugates if and only if, when written as a composition of disjoint cycles (in which every number appears, i.e. counting $1$-cycles), they have the same number of cycles of each length.
\end{corollary}
\begin{proof}
    Follows directly from the formula in the preceding proposition and Theorem \ref{disjoint_cycles}.
\end{proof}
For a permutation, we can produce an (unique) string $1^{a_1}2^{a_2}\cdots n^{a_n}$ where $a_i$ is the number of cycles of length $i$.
We call such a string the ``cycles type'' of a permutation.
So the above corollary means that two permutations are in the same conjugacy class if and only if they have the same cycle type.
It is then curious to consider the size of each conjugacy class.
\begin{definition}
    The stabiliser of an element $g$ under the conjugacy action is the centraliser, written as $C_G(g)$.
\end{definition}
\begin{lemma}
    If $\tau\in S_n$ has cycle type $1^{a_1}2^{a_2}\cdots n^{a_n}$, then
    $$|C_G(\tau)|=1^{a_1}(a_1)!2^{a_2}(a_2)!\cdots n^{a_n}(a_n)!$$
\end{lemma}
\begin{proof}
    Obvious.
\end{proof}
\begin{corollary}
    The size of the conjugacy class containing $\tau$ is
    $$\frac{n!}{1^{a_1}(a_1)!2^{a_2}(a_2)!\cdots n^{a_n}(a_n)!}$$
    where $a_i$ are defined as before.
\end{corollary}
\begin{proof}
    Orbit-Stabiliser.
\end{proof}
\begin{example}
    Consider $S_4$, then the conjugacy classes are of sizes $1$ (consisting of $e$ only and have cycle type $1^4$), $6$ (of type $1^22^1$), $3$ (of type $2^2$), $8$ (of type $1^13^1$), and $6$ (of type $4^1$) by the formula.
    We do have $1+6+3+8+6=24=4!=|S_4|$.
    Note that given the number of conjugacy classes that we expect, it is trivial to work out what are the elements.
\end{example}
\begin{corollary}
    Any normal subgroup of a finite group $G$ is an union of conjugacy classes of $G$.
\end{corollary}
\begin{proof}
    If $h\in H$ is in one of the conjugacy class, then by normality $ghg^{-1}\in H$ for any $g$, so the entire conjugacy class is in $H$.
\end{proof}
\begin{example}
    We try to find the normal subgroups of $S_4$.
    Let $H\unlhd S_4$, then $H$ must contain the conjugacy class $\{e\}$.
    If $H$ contains the conjugacy class $1^22^1$, then since transpositions generates $S_4$, $H$ is the entire group $S_4$.\\
    If $H$ contains the conjugacy class $2^2$, then it contains the normal subgroup $K$ consisting of the conjugacy classes $1^4,2^2$ only.
    We the case $H>K$ means that $H$ contains more than $2$ conjugacy classes, so we can discuss this case later by considering other conjugacy classes.\\
    If $H$ contains the conjugacy class $3^1$, then it contains all $3$-cycles, which there are $8$ of them, so $|H|\ge 9$, so we must have $|H|=12$ or $24$.
    Note that $3$-cycles are even, so $H\cap A_4$ contains all $3$-cycles, which have at least $9$ elements, thus $H\cap A_4=A_4$, so $H=A_4$ or $H=S_4$.\\
    If $H$ contains the conjugacy class $4^1$, but then it contains at least one $3$-cycles (e.g. $(1\ 2\ 3\ 4)(1\ 4\ 2\ 3)=(2\ 4\ 3)$), but since $H$ is normal it in fact contains all $3$-cycles, then it is just the same as the previous case (where, since $(1\ 2\ 3\ 4)$ is odd, we have $H=S_4$).\\
    So $H$ is one of $K,A_4,S_4$.\\
    We have $S_4/S_4\cong \{e\},S_4/A_4\cong C_2, S_4/K\cong S_3$.
\end{example}
As $A_n\unlhd S_n$, for $\sigma\in A_n$, $\operatorname{ccl}_{A_n}(\sigma)\subseteq\operatorname{ccl}_{S_n}(\sigma)$, but the equality may not hold.
For example, $(1\ 2\ 3)$ and $(1\ 3\ 2)$ are even and conjugates of each other in $S_3$, but they are not in $A_3$ which is abelian.
On the other hand, in $S_5$, we have, however, $[(2\ 3)(4\ 5)](1\ 2\ 3)[(2\ 3)(4\ 5)]^{-1}=(1\ 3\ 2)$, and $(2\ 3)(4\ 5)$ is even, so they are conjugate in $A_5$ in this case.\\
For $\sigma\in A_n$, we have
$$|A_n|/|\operatorname{ccl}_{A_n}(\sigma)|=|C_{A_n}(\sigma)|,|S_n|/|\operatorname{ccl}_{S_n}(\sigma)|=|C_{S_n}(\sigma)|$$
But we also have $|S_n|=2|A_n|$, so either the conjugacy classes are the same, which implies $|C_{A_n}(\sigma)|=|C_{S_n}(\sigma)|/2$.
Otherwise, $|\operatorname{ccl}_{A_n}(\sigma)|=|\operatorname{ccl}_{S_n}(\sigma)|/2$ and $C_{A_n}(\sigma)=C_{S_n}(\sigma)$.\\
Thus either the centraliser of $\sigma$ contains an odd element and the conjugacy classes are the same or the centralisers of $\sigma$ is contained in $A_n$ and the conjugacy class in $A_n$ is half the size of that in $S_n$.
\begin{example}
    Consider $S_4$ and $A_4$.
    We look at the conjugacy classes in $S_4$ and study whether they split in $A_4$.
    $\{e\}$ itself constitutes a conjugacy class, so there is nothing to show.
    The conjugacy class $1^22^1$ of transpositions is all odd, thus does not lie in $A_4$.
    $2^2$ in $S_4$ are double transpositions, which are even, so it do lie in $A_4$, but there are only $3$ elements, so it cannot split.
    On the other hand, $(1\ 2)$ centralises $(1\ 2)(3\ 4)$, so the centraliser does contain an odd element.\\
    The conjugacy class $1^13^1$ are even so do lie in $A_4$.
    And the centraliser of it is contained in $A_4$, i.e. all even elements, so the conjugacy class splits into two.
    Indeed, they splits to give $\{(1\ 2\ 3),(1\ 4\ 2),(1\ 3\ 4),(2\ 4\ 3)\}$ and $\{(1\ 3\ 2),(1\ 2\ 4), (1\ 4\ 3),(2\ 3\ 4)\}$.\\
    The $4$-cycles in $S_4$ are odd, so do not lie in $A_4$, hence again there is nothing to show.
\end{example}
We can also use it to search for normal subgroups of $A_4$.
\begin{example}
    By definition a normal subgroup of $A_4$ must be the union of conjugacy classes.
    Then we could either have $\{e\}$, or $K$, constituting of $\{e\}$ with all the double transpositions.
    Note that if the normal subgroup contains one of the $1^13^1$ conjugacy classes, it must contain the other one which constitutes the inverses of it.
    Hence it must contain at least $9$ elements, but $|A_4|=12$, so it can only be $A_4$.
    Therefore the normal subgroups are $\{e\},K,A_4$.
\end{example}
In particular, $A_4$ is not simple.
\begin{theorem}
    $A_5$ is simple.
\end{theorem}
In fact, $A_n$ is simple for any $n\neq 4$.
\begin{proof}
    $S_5$ has $120$ elements, and its conjugacy classes can be summarized as $1^5,1^32^1,1^12^2,1^23^1,2^13^1,1^14^1,5^1$, we can have the following table
    \begin{center}
        \begin{tabular}{c|ccccccc}
            Cycle type&$1^5$&$1^32^1$&$1^12^2$&$1^23^1$&$2^13^1$&$1^14^1$&$5^1$\\
            Size&$1$&$10$&$15$&$20$&$20$&$30$&$24$\\
            Sign&+&-&+&+&-&-&+
        \end{tabular}
    \end{center}
    By looking at whether the conjugacy class split for a typical even permutation of each cycle type, we conclude the following sizes of conjugacy classes in $A_5$.
    \begin{center}
        \begin{tabular}{c|ccccc}
            Cycle type&$1^5$&$2^2$&$3^1$&$5^1$&$5^1$\\
            Size&$1$&$15$&$20$&$12$&$12$
        \end{tabular}
    \end{center}
    Thus there is no way to sum them up (where we must of course add the first cycle type that is the identity) to produce a factor of $|A_5|=60$.
    Therefore $A_5$ is simple.
\end{proof}