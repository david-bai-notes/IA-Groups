\section{The Isomorphism Theorems}
\begin{definition}
    A subgroup $H\le G$ is called normal if $\forall h\in H,\forall g\in G, ghg^{-1}\in H$, in which occasion $H\unlhd G$
\end{definition}
\begin{example}
    1. $\{e\}\unlhd G,G\unlhd G$.\\
    2. The subgroup of the dihedral group $D_{2n}$ generated by the rotations is normal. but that generated by the reflection generator is not normal (given $n\ge 3$).\\
    3. If $G$ is abelian, then for every $H\le G,H\unlhd G$.
\end{example}
\begin{lemma}
    A subgroup $H\le G$ is normal if and only if $\forall a\in G,aH=Ha$.
\end{lemma}
\begin{proof}
    Trivial but let us write it down.\\
    If $H$ is normal, then for any $ah\in aH,\exists h'\in H,aha^{-1}=h'\implies ah=h'a\in Ha$, so $aH\subset Ha$.
    Similarly $Ha\subset aH$, so $Ha=aH$.\\
    Conversely, if $\forall a\in G,Ha=aH$, we can choose any $h\in H,\exists h'\in H,ah=h'a\implies aha^{-1}=h'\in H$, so $H$ is normal.
\end{proof}
\begin{corollary}
    Let $H\le G$.
    If $|G/H|=2$, then $H$ is normal.
\end{corollary}
\begin{proof}
    If $a\in H$, then obviously $aH=Ha$, otherwise, since $H$ has index $2$, $aH=G\setminus H=Ha$, thus $H\unlhd G$.
\end{proof}
\begin{proposition}
    Let $\phi:G\to K$ be a homomorphism, then $\ker\phi\unlhd G$.
\end{proposition}
\begin{proof}
    Suppose $h\in\ker\phi$, then $\forall g\in G$, then $\phi(ghg^{-1})=\phi(g)e_K\phi(g)^{-1}=e_K\implies ghg^{-1}\in\ker\phi$.
\end{proof}
So we know that every kernel is a normal subgroup, but how about the converse?
Must every normal subgroup the kernel of some homomorphism?
\begin{definition}
    Let $G$ be a group and $H\unlhd G$, then we can define the operation
    $$(aH)\cdot(bH)=(ab)H$$
    And $G/H$ is a group under this operation.
    This is called the quotient group.
\end{definition}
Given that it is well defined, which we will prove later, then we know that whenever $H$ is normal, then it is the kernel of the homomorphism $\pi: G\to G/H$ by $\pi(a)=aH$.
This is called the caconical projection.\\
If we do not have $H$ being normal, then if we want to define the operation
$$aH\times bH=abH$$
But is it well defined?
Note that to do so, if $aH=a'H,bH=b'H$, then $a^{-1}a',b^{-1}b'\in H$, but to make the operation well-defined, we must have
$$a'b'H=abH\iff a^{-1}b^{-1}a'b'\in H$$
But this is not always true.
But if $H$ is normal, then it is however true.
In fact, this is true if and only if $H$ is normal.
\begin{theorem}
    Our operation on quotient group is well-defined and $G/H$ is a group under it.
\end{theorem}
\begin{proof}
    If $aH=a'H,bH=b'H$, then
    $$a'b'H=a'bH=a'Hb=aHb=abH$$
    Due to normality of H.\\
    To see that $G/H$ is thus a group, we observe that $(aH\cdot bH)\cdot cH=abcH=aH\cdot(bH\cdot cH)$.
    Also $H=eH$ is the identity and $gH\cdot g^{-1}H=H$.
\end{proof}
\begin{example}
    1. Note that $n\mathbb Z\le\mathbb Z$, and it is normal since $\mathbb Z$ is abelian.
    Now $\mathbb Z_n\cong \mathbb Z/n\mathbb Z$ by the isomorphism $k\mapsto k+n\mathbb Z$.\\
    2. Let $R=\langle r|r^n\rangle\le D_{2n}$, then $|D_{2n}/R|=2$, so $D_{2n}/R\cong C_2$.\\
    3. Let $K$ be the group consisting of $\{e,r^2\}$ in $D_8$.
    One can check that this is normal, and that $D_8/K\cong K_4=C_2\times C_2$ since (by inspection) every element in the quotient group has order $2$.\\
    4. Let $K$ be the subgroup of $Q_8$ consisting of $\{\pm\underline{1}\}$, and $Q_8/K\cong K_4$ since every element has order $2$ again.
    From this example and the last one, we can see that if $H_1\le G_1,H_2\le G_2$ and $H_1\cong H_2,G_1/H_1\cong G_2/H_2$, we do not necessarily have $G_1\cong G_2$.
    Hence, when we dissolve a group into normal subgroup and quotient, there might not be an unique way to rebuild the group from them.
\end{example}
\begin{theorem}[First Isomorphism Theorem]\label{1_isom_thm}
    Suppose $\phi:G\to H$ is a homomorphism, then $G/\ker\phi\cong\operatorname{Im}\phi$.
    Indeed, the map $\bar\phi:G/\ker\phi\to\operatorname{Im}\phi$ by $g\ker\phi\mapsto\phi(g)$ is well defined and is an isomorphism.
\end{theorem}
The theorem gives the following commutative diagram, where $\pi$ is the caconical projection:
$$
\begin{tikzcd}
    G\arrow{r}{\phi} \arrow[swap]{d}{\pi} & \operatorname{Im}\phi\\
    G/\ker\phi\arrow[swap,dashed]{ur}{\bar\phi}&
\end{tikzcd}
$$
\begin{proof}
    We know that $\ker\phi$ is normal, thus we can form the quotient $G/\ker\phi$.\\
    If $g\ker\phi=h\ker\phi$, then $h^{-1}g\in\ker\phi$, hence
    $$e_H=\phi(h^{-1}g)=\phi(h)^{-1}\phi(g)\implies \phi(g)=\phi(h)$$
    thus $\bar\phi$ is well-defined.
    Note also that
    $$\bar\phi((g\ker\phi)(h\ker\phi))=\bar\phi(gh\ker\phi)=\phi(gh)=\phi(g)\phi(h)=\bar\phi(g\ker\phi)\bar\phi(h\ker\phi)$$
    so it is a homomorphism.
    Furthermore, if $\bar\phi(g\ker\phi)=\bar\phi(h\ker\phi)$, then
    $$\phi(g)=\phi(h)\implies \phi(h^{-1}g)=e_H\implies h^{-1}g\in\ker\phi\implies h\ker\phi=g\ker\phi$$
    So it is injective.
    It is also surjective by definition, so it is bijective, hence it is an isomorphism.
\end{proof}
\begin{example}
    1. Consider $\mathbb Z_n\cong\mathbb Z/n\mathbb Z$.
    Now an easy proof of that is to recognize the homomorphism $\mathbb Z\to\mathbb Z_n$ sending an integer to the remainder it left when divided by $n$.
    And the result follows by Theorem \ref{1_isom_thm}.\\
    2. The function $\phi:(\mathbb R,+,0)\to(\mathbb C\setminus\{0\},\times,1)$ by $t\mapsto e^{2\pi it}$, so the image of $\phi$ is $S^1=\{z\in\mathbb C:|z|=1\}$ and $\ker\phi=\mathbb Z$, so $\mathbb R/\mathbb Z\cong S^1$.\\
    3. If $H$ and $G$ are groups, we have $G\times H$ and $\{e\}\times H\unlhd G\times H$, but $G\times H/\{e\}\times H\cong G$.\\
    4. Let $G$ be the group of all symmetries (isometries) of the tetrahedron, then consider the map $\phi:G\to\operatorname{Sym} V$ where $V$ is the set of vertices by action.
    But this is injective and $\operatorname{Sym}V\cong S_4$ has $24$ elements, and the rotational symmetries forms an order-$12$ proper subgroup of $G$, thus we must have $G\cong G/\{e\}\cong\operatorname{Im}\phi\le S_4$, but $12<|\operatorname{Im}\phi||24=|S_4|$ by Theorem \ref{1_isom_thm}, so by Corollary \ref{lagrange}, $\operatorname{Im}\phi\cong S_4$.
    5. $G$ also acts on the opposite pairs of edges, which has order $3$, so it gives a homomorphism $\phi:G\to S_3$.
    Since its image has an element of order $2$ and an element of order $3$, this homomorphism is surjective, therefore $S_4/\ker\phi=G/\ker\phi\cong S_3$, so $|\ker\phi|=4$.
    Interestingly, there is never again a surjective homomorphism from $S_n$ to $S_{n-1}$ for $n>4$.
\end{example}
\begin{definition}
    A group $G$ is simple if it has no proper normal subgroup.
\end{definition}
\begin{example}
    $C_p$ is simple for $p$ prime, since it does not even have any proper subgroup by Corollary \ref{lagrange}.
\end{example}
