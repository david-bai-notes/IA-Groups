\section{The M\"obius Group}
We want to study $f:\mathbb C\to\mathbb C$ in the form 
$$f(x)=\frac{ax+b}{cx+d},a,b,c,d\in\mathbb C$$
This function has a pole at $x=-d/c$, so we need an element at infinity.
We can take $\mathbb C_\infty:=\mathbb C\cup\{\infty\}$ by the stereographic projection $\mathbb C_{\infty}=\mathbb C\cup\{\infty\}\cong S^2$.
Now we define the Mobius map properly
\begin{definition}
    The Mobius map $f:\mathbb C_\infty\to\mathbb C_\infty$ is defined by
    $$
    f(z)=
    \begin{cases}
        \frac{az+b}{cz+d}\text{, if $z\neq\infty$ and $z\neq -d/c$}\\
        \infty\text{, if $z=-d/c$}\\
        \frac{a}{c}\text{, if $z=\infty$}
    \end{cases}
    $$
    where $ad-bc\neq 0$.
\end{definition}
The reason why we impose the last condiciton is that we want the Mobius map to be a bijection from $\mathbb C_\infty$ to $\mathbb C_\infty$.
\begin{proposition}
    Let $\mathcal M=\{f:\mathbb C_\infty\to\mathbb C_\infty:f\text{ is a Mobius function.}\}$.
    Then $(\mathcal M,\circ,\mathrm{id})$ is a group.
\end{proposition}
\begin{proof}
    Obviously $\mathrm{id}\in\mathcal M$.
    Note also that if $g(z)=(az+b)/(cz+d), g'(z)=(a'z+b')/(c'z+d')$, then $g'(g(z))=(a''z+b'')/(c''z+d'')$ where
    $$
    \begin{pmatrix}
        a''&b''\\
        c''&d''
    \end{pmatrix}
    =
    \begin{pmatrix}
        a'&b'\\
        c'&d'
    \end{pmatrix}
    \begin{pmatrix}
        a&b\\
        c&d
    \end{pmatrix}
    $$
    Then it immediately tells us that $\mathcal M$ is closed under $\circ$ since the determinant function is multiplicative.
    Note as well that it also gives us the inverse by just finding some $a',b',c',d'$ (which exists due to our criterion on determinant) such that
    $$
    \begin{pmatrix}
        a'&b'\\
        c'&d'
    \end{pmatrix}
    =
    \begin{pmatrix}
        a&b\\
        c&d
    \end{pmatrix}^{-1}
    $$
    by noticing that the identity function corresponds to $cI,c\neq 0$.
\end{proof}
We can have $\mathcal M$ to act on $\mathbb C^\infty$ faithfully (with trivial kernel), so $\mathcal M\le\operatorname{Sym}\mathbb C_\infty$.
Now consider the Mobius transformation $f(z)=1/(z-a)$, which sends $a$ to $\infty$ and its inverse that sends $\infty$ to $a$, so there is nothing special with $\infty$ in $\mathbb C^\infty$, as one will expect as there is no special point on $S^2$.
\begin{proposition}\label{decomp_mobius}
    Every Mobius tranformation is a composition of $z\mapsto az,a\neq 0$, $z\mapsto z+b$ and $z\mapsto 1/z$.
\end{proposition}
\begin{proof}
    Let $z\mapsto (az+b)/(cz+d)$ be a mobius transformation, then if $c=0$ the proposition is trivial.
    Otherwise $c\neq 0$, then we have
    $$\frac{az+b}{cz+d}=\frac{a}{c}-\frac{ad-bc}{c(cz+d)}$$
    which can obviously be obtained from the said functions.
\end{proof}
Now, how about fixed point of a Mobius transformation?
We know that a Mobius transformation fixes at least $1$ point, but how about more?
\begin{proposition}
    A Mobius transformation fixes $3$ points is the identity.
\end{proposition}
\begin{proof}
    Suppose
    $$f:z\mapsto\frac{az+b}{cz+d}$$
    If $\infty$ is a fixed point, then $c=0$, so $f$ is a linear function.
    But then a linear function that fixes $2$ (non-infinity) points is the identity (since $f(x)-x$ is linear and a linear function has exactly $1$ root unless it is constantly $0$), $f$ is the identity.\\
    Now if $\infty$ is not a fixed point, then
    $$f(z)=z\iff \frac{az+b}{cz+d}-z=0\iff az+b-z(cz+d)=0$$
    which has at most $2$ roots (hence fixed point) since it is quadratic, unless it is the zero function, which essentially means that $c=b=0,d=a\neq 0\implies f=\mathrm{id}$.
\end{proof}
\begin{proposition}\label{mobius_3pts}
    Given distinct $z_1,z_2,z_3\in\mathbb C_\infty$ and $w_1,w_2,w_3\in\mathbb C$, then there is an unique Mobius transformation $f$ such that $f(z_i)=w_i$ for $i\in\{1,2,3\}$.
\end{proposition}
Note that since every Mobius transformation is a bijective (hence invertible), $w_i$'s are distinct as well.
\begin{proof}
    For existence, it suffices to deal with the case where $w_1,w_2,w_3$ are $0,1,\infty$, since once we've found maps $f,g$ such that $f:z_1,z_2,z_3\mapsto 0,1,\infty,g:w_1,w_2,w_3\mapsto 0,1,\infty$, then $g^{-1}\circ f$ will send $z_1,z_2,z_3$ to $w_1,w_2,w_3$.\\
    Now if none of $z_i$'s is $\infty$, we can use the interpolation
    $$f(z)=\frac{(z-z_2)(z-z_3)}{(z_1-z_2)(z_2-z_3)}+\frac{(z-z_1)(z-z_2)}{z-z_3}$$
    Otherwise, suppose $z_i=\infty$, then the map $f_i$ suffices where
    $$f_1(z)=\frac{z-z_2}{z-z_3},f_2(z)=\frac{z_1-z_3}{z-z_3}, f_3(z)=\frac{z-z_2}{z_1-z_2}$$
    For uniqueness, suppose $f,f'$ send $z_1,z_2,z_3$ to $w_1,w_2,w_3$ respectly, then $f^{-1}\circ f'$ fixes $z_1,z_2,z_3$, hence $f^{-1}\circ f'=\mathrm{id}\implies f=f'$.
\end{proof}
If $f,g\in\mathcal M$ and $f$ fixes $z_0$, then $gfg^{-1}$ fixes $g(z_0)$, which gives rise to the following observation
\begin{theorem}
    %There are three conjugacy classes of $\mathcal M$, namely the identity alone, the functions with exactly one fixed point, and the functions with exactly $2$ fixed points.
    Every member of a conjugacy class of $\mathcal M$ has the same number of fixed point(s).
\end{theorem}
\begin{proof}
    Obviously the identity itself is itself a conjugacy class.
    Now for any nonidentity $f$, $f$ has either $1$ or $2$ fixed points.\\
    If $f$ has $1$ fixed point $z_0\neq\infty$, then suppose $g(z)=1/(z-z_0)$, we know that $gfg^{-1}$ fixes $\infty$, and it cannot fix any other points because if so then applying $g^{-1}$ to that point would produce another fixed point of $f$.
    So it has to be the map $z\mapsto z+b,b\neq 0$.\\
    If $f$ has $2$ fixed point, then we consider a Mobius transformation $g$ which sends the fixed points to $0,\infty$, then $gfg^{-1}$ fixed $0$ to $\infty$ and sends $1$ to $a\in\mathbb C\neq 0,\infty$, so there is exactly one Mobius transformation $z\mapsto az, a\neq 1$.
\end{proof}
Note that $(g^{-1}fg)^n=g^{-1}f^ng$.
This allows us to compute the arbitrary (integral) power of a Mobius transformation.
\begin{definition}
    The circle in the extended complex numbers is the equation $Az\bar z+\bar Bz+B\bar z+C=0$ with $A,C\in\mathbb R,B\in\mathbb C$.
    Consider $\infty$ is a point on this circle if and only if $A=0$.
\end{definition}
Note that circles in $\mathbb C$ are also circles in $\mathbb C_\infty$, and all other $\mathbb C_\infty$ circles are lines in $\mathbb C$.
\begin{proposition}
    Circles in $\mathbb C_\infty$ are mapped to circles in $\mathbb C_\infty$ under any Mobius transformations.
\end{proposition}
\begin{proof}
    It is sufficient to verify this for $z\mapsto az,z\mapsto z+b,z\mapsto z^{-1}$ due to Proposition \ref{decomp_mobius}.
    It is then trivial.
\end{proof}
Note that every circle gets to mapped to any other circle since three points determine the circle and Proposition \ref{mobius_3pts}
\begin{definition}
    For extended complex numbers $z_1,z_2,z_3,z_4$, the cross ratio is defined by
    $$[z_1,z_2,z_3,z_4]=\frac{(z_4-z_1)(z_2-z_3)}{(z_2-z_1)(z_4-z_3)}$$
\end{definition}
We need to examine carefully when one of these numbers is infinity.
For example, if we have $z_1=\infty$, then the cross ratio is $(z_2-z_3)/(z_4-z_3)$.
\begin{corollary}
    For extended complex numbers $z_1,z_2,z_3,z_4$, the cross ratio is equal to $f(z_4)$ where $f$ is the unique Mobius transformation sending $z_1,z_2,z_3$ to $0,1,\infty$ respectively.
\end{corollary}
\begin{theorem}
    Mobius transformations preserve the cross-ratio.
\end{theorem}
\begin{proof}
    Suppose $z_1,z_2,z_3,z_4\in\mathbb C_\infty$, and $g\in\mathcal M$.
    Let $f$ be the Mobius transformation sending $z_1,z_2,z_3$ to $0,1,\infty$, so the cross ratio is $f(z_4)$, so $f\circ g^{-1}$ sends $g(z_1),g(z_2),g(z_3)$ to $0,1,\infty$, so the cross ratio of the $g(z_i)$'s is $f\circ g^{-1}(g(z_4))=f(z_4)$.
\end{proof}
The converse is also true (and proved in example sheet): If a map preserves cross-ratio, then it is a Mobius transformation.
\begin{corollary}
    Four points $z_1,z_2,z_3,z_4\in\mathbb C_\infty$ are on a circle (in the sense of $\mathbb C_\infty$) if and only if $[z_1,z_2,z_3,z_4]$ is real.
\end{corollary}
\begin{proof}
    Let $f$ be the unique Mobius transformation sending $z_1,z_2,z_3$ to $0,1,\infty$, so $[z_1,z_2,z_3,z_4]=f(z_4)$.
    Let $c$ be the unique circle passing through $z_1,z_2,z_3$, then $z_4\in c\iff f(z_4)\in f(c)$, but $f(c)=\mathbb R\cup\{\infty\}$.
\end{proof}