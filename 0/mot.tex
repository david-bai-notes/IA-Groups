\section{Motivation of Group}
Groups are the mathematical notion of symmetries.
Indeed, if we go further into this topic, we will find that the symmetries of anything give a group, and any group is actually the symmetry of something.
Then why study symmetries in this way?
Why do we not just desribing the symmetries one by one instead?\\
Consider a tetrahedron.
There are $12$ rotational symmetries of it: $1$ doing nothing, $8$ rotations on an axis joining one of the vertices and the centre of the tetrahedron, and $3$ rotations on the axis joining the midpoints of opposite sides.
The interesting thing is, if you composite two of the rotations that are listed above, you get another rotation.
For example, if we label the vertices as $1,2,3,4$, then one of the rotations on the axis passing though vertex $1$ may send the vertices like
$$1\to 1, 2\to 4, 3\to 2, 4\to 3$$
and the rotation on the axis joining midpoints of opposite segments would permute them by
$$1\to 3, 2\to 4, 3\to 1, 4\to 2$$
We let $R$ be the former rotation and $S$ be the latter, then we could find $S\circ R$, the rotation given by doing $R$ first then $S$ next.
Indeed, this is the permutation
$$1\to 3, 2\to 2, 3\to 4, 4\to 1$$
which is one of the rotations on the axis passing though $2$.
Here is an interesting thing: we can do $R\circ S$ as well, but it is, as one can check, a rotation on the axis passing though $4$!
So $RS\neq SR$, as in the order of the composition of two rotations matters.
This is kind of the point of group theory.\\
Now we can look at the rotational symmetries of another soli: an isocagonal cone.
This time it is quite obvious: the rotational symmetries are precisely the rotations on the central axis of degree $n\pi/6$ where $n=0,1,2,\ldots, 11$.
Now this set of rotations has order $12$ as well.
But are the two sets of rotational symmetries, one on a tetrahedron and another on a isocagonal cone, the same?
No, our intuition said. But why?
\begin{proposition}
    The groups of rotational symmetries of a tetrahedron and a isocagonal cone are different.
\end{proposition}
\begin{proof}
    Note that every rotation in the tetrahedron group is the same as doing nothing after repeating itself 2 or 3 (so, 6) times. 
    But the rotation of $\pi/6$ degrees of the isocagonal cone does not have this property.
    Therefore the two groups are different.
\end{proof}
There is another way of doing it,
\begin{proof}[Alternative proof]
    In our previous example, we have found two rotations $S$ $R$ such that they do not commute, i.e. $RS\neq SR$.
    However, every two rotations in the isocagonal cone group commmutes.
    Therefore they are different.
\end{proof}
Note that in the second proof, there is an important property of groups that was used: commutativity.
This notion, expressed in several contexts, is essential to group theory.
But to see that, you have to dive into the world of groups.
