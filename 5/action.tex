\section{Group Actions}
\begin{definition}
    Let $(G,\cdot,e)$ be a group and $S$ be a set.
    The action of $G$ on $S$ is a function $\star:G\times S\to S$ satisfying\\
    A1. $\forall x\in S,e\star x=x$.\\
    A2. $\forall g_1,g_2\in G, \forall x\in S, (g_1\cdot g_2)\star x=g_1\star(g_2\star x)$.
\end{definition}
\begin{example}
    1. $\forall g\in G, x\in S, g\star x=x$ always defines a group action.\\
    2. A group $G$ acts on the set $S=G$ by left multiplication.
    That is, via $g\star g'=g\cdot g'$.
    This is called the \textit{left regular action}.\\
    3. Consider $\operatorname{Sym}S$, it acts on $S$ by applying the permutation.\\
    4. The symmetries of a solid $X$ acts on the set of points of $X$ (or a special subset like the set of vertices).
    Note that the dihedral group $D_{2n}$ acts on an $n$-gon in this way.
\end{example}
\begin{definition}
    The orbit of $x\in S$ is the set
    $$G\star x=\{y\in X:\exists g\in G,y=g\star x\}$$
    If this set is equal to $X$, then this action is called transitive.
    The stabiliser of $x\in S$ is the set
    $$G_x=\{g\in G:g\star x=x\}$$
    The kernel is defined as
    $$\{g\in G:\forall x\in S,g\star x=x\}=\bigcap_{x\in S}G_x$$
    An action is faithful if its kernel is $\{e\}$.
\end{definition}
\begin{theorem}
    An action is the same as a homomorphism $\rho:G\to\operatorname{Sym}S$.
\end{theorem}
\begin{proof}
    The function $t_g:x\mapsto g\star x$ is a permutation of $X$ for any $g\in G$.
    Indeed, we can find an inverse of it which is exactly $t_{g^{-1}}$.
    Now we shall show that the map $\rho:g\mapsto t_g$ is an homomorphism.
    We can evaluate $\rho(gh)=t_{gh}$, but $t_{gh}(x)=(gh)\star x=g\star(h\star x)=t_g\circ t_h(x)$, therefore
    $$\rho(gh)=t_g\circ t_h=\rho(g)\rho(h)$$
    Conversely, let $\rho$ be an homomorphism, then consider the function $\star$ defined by $g\star x=(\rho(g))(x)$ is an action.
\end{proof}
Note that the same actions are corrsponded to the same homomorphism.
\begin{theorem}[Cayley's Theorem]
    Any group is isomorphic to a subgroup of some symmetric group.
\end{theorem}
\begin{proof}
    Consider the left regular action of the group $G$ on the set $G$.
    There is then a homomorphism $\rho:G\to \operatorname{Sym}G$.
    Now this action is faithful.
    Indeed, $\rho(g)=e\iff\forall x\in G, gx=x\iff g=e$ (Or simply we can write $g=ge=e$).\\
    Therefore this homomorphism is injective, hence $G\cong\operatorname{Im}\rho\le\operatorname{Sym}G$.
\end{proof}
We now want to dive deeper into orbits and stabiliser.
Let $G$ act on $X$.
\begin{theorem}
    For each $x\in X$, $G_x\le G$, and the collection of all orbits $G\star x$ for every $x\in X$ partitions $X$.
\end{theorem}
\begin{proof}
    Note that $e\in G_x$ for each $x$ so $G_x$ is nonempty.
    If $a,b\in G_x$, then $x=e\star x=(b^{-1}b)\star x=b^{-1}\star (b\star x)=b^{-1}\star x\implies b^{-1}\in G_x$.
    Therefore $(ab^{-1})\star x=a\star(b^{-1}\star x)=a\star x=x\implies ab^{-1}\in G_x$, so $G_x\le G$.\\
    Now the union of orbits contains each $x\in X$ since $x\in G\star x$.
    Now if $G\star x\cap G\star y\neq\varnothing$, there is some $g,h\in G$ such that $g\star x=h\star y$.
    Therefore for any $z\in G\star x$, then $z=k\star x$ for some $k\in G$, then $z=k\star((g^{-1})\star(h\star y))=(kg^{-1}h)\star y\implies z\in G\star y$.
    So $G\star x\subseteq G\star y$ and $G\star y\subseteq G\star x$, so $G\star x=G\star y$.
    Hence the orbits form a parition of $X$.
\end{proof}
What we have discussed so far is called the left action.
The right action $\diamond:X\times G\to X$ is defined analogously.
So $x\diamond e=x$ and $(x\diamond g)\diamond h=x\diamond (gh)$.
\begin{proposition}
    If $\diamond$ is a right action, then we can define a left action $\star$ by $g\star x=x\diamond g^{-1}$.
    We can also have the other way around.
\end{proposition}
\begin{proof}
    Trivial.
\end{proof}
So we can only use left actions during the scope of our study.
\begin{definition}
    If $G$ has a left action on $X$, the set of orbits is called $G\backslash X=\{G\star x:x\in X\}$.
    If $G$ has a right action on $X$, the set of orbits is called $X/G=\{x\diamond G:x\in X\}$
\end{definition}
\begin{example}
    The dihedral group $D_{2n}$ acts on the regular $n$-gon $X$ in the obvious way.\\
    The orbit of a vertex is then all vertices.
    The stabiliser of a vertex is the identity and the reflection across the axis which is the diagonal through the vertex.
    Note here that the size of the orbit times the size of stabiliser is the size of the dihedral group.\\
    Now consider a point in the interior of a side of the square.
    Then the orbit of this point would consist of $8$ points, and nothing stabilises it, so we also get that the product of the sizes of the orbit and the stabiliser is $8$, the size of the dihedral group.\\
    Let us now look at the symmetries of the tetrahedron with vertices $1,2,3,4$.
    Suppose the rotation across midpoints of opposite sides be $S$ and the rotation across the central axis through a vertex $R$.\\
    Let $V$ be the set of vertices and let this group act on it.
    Then it is obvious that this action is transitive, and the stabiliser of a vertex are the rotations with respect to the axis through that vertex.\\
    So again the sizes of the orbit and stabiliser give a product of $12$, the size of the group.\\
    Now we act on the set of edges $E$.
    Pick one of the edge $X$, then the action of the group on it is yet again transitive, and the stabilisers are the identity and the rotation $R$ on the midpoint of $E$ and its opposite edge.
    Again they give a product of $6\times 2=12$, the size of the group.
\end{example}
The preceding observation triggers the following theorem.
\begin{theorem}[Orbit-Stabiliser Theorem on finite groups]\label{ost_finite}
    $|G_x||G\star x|=|G|$
\end{theorem}
To prove it, we need some preparation.
\begin{proposition}
    If $H\le G$, the left regular action of $H$ on $G$ is the left multiplication of element in $H$ on $G$, i.e. $h\star g=hg$.
    The right regular action then is $g\diamond h=gh$
\end{proposition}
\begin{definition}
    A left coset of $H$ in $G$ is an orbit of the right regular action.
    Write $G/H$ to denote the collection of these cosets.\\
    We can define the right coset the other way around which are collected as $H\backslash G$.
\end{definition}
So each left coset is in the form $gH=\{gh:h\in H\}$, and the right coset in the form $Hg=\{hg:h\in H\}$.
Note that it is not always true that a left coset is equal to the right coset.\\
Also $G/H=\{S\subseteq G:\exists g\in G,S=gH\}$ and we can find $H\backslash G$ similarly.
\footnote{Some authors use $G:H$ for the collection of left cosets.}\\
Note that $gH=g'H\iff g'^{-1}g\in H$, and for right cosets $Hg=Hg'\iff g'g^{-1}\in H$ but again these two conditions may or may not be equivalent.
\begin{example}
    The (left and right) cosets of $2\mathbb Z$ are $2\mathbb Z$ and $2\mathbb Z+1=1+2\mathbb Z$.\\
    The (left and right) cosets of $n\mathbb Z$ are $k+n\mathbb Z, k\in\{0,1,\ldots n-1\}$.\\
    Consider $D_6$, $R=\{e,r,r^2\}, S=\{e,s\}$ are subgroups.
    The (left and right) coset of $R$ are $R$ and $sR=Rs$
    And the left cosets of $S$ are $S$, $rS$ and $r^2S$ and the right ones are $S$, $Sr$, $Sr^2$, but in this case the left and right cosets are not equal.
\end{example}
\begin{theorem}
    If $H\subset G$ and $g\in G$, then there is a bijection $H\to gH$.
\end{theorem}
\begin{proof}
    $h\mapsto gh$ is the bijection.
\end{proof}
\begin{corollary}[Lagrange's Theorem]\label{lagrange}
    If $G$ is finite and $H\le G$, then we have $|H||G/H|=G$.
    In particular $|H|$ divides $|G|$.
\end{corollary}
\begin{proof}
    The cosets are orbits thus parition $G$ and they are of the same cardinality due to the preceding theorem.
\end{proof}
We can repeat the same argument to see that the same also applies to right cosets.
\begin{definition}
    Let $G$ be a group, and $H\le G$, then the index of $H$ is the order of $G/H$ given that it is finite, otherwise we say the index is infinite.
\end{definition}
Equivalently, the index is $|G|/|H|$ (given that $H$ is finite).
\begin{corollary}
    If $G$ is finite and $g\in G$, then $g^{|G|}=e$, that is, $\operatorname{ord}g||G|$.
\end{corollary}
\begin{proof}
    Consider the subgroup $\langle g\rangle\le G$.
\end{proof}
\begin{corollary}
    If $G$ is finite $|G|$ is prime, then $G\cong C_p$ and it is generated by any identity element.
\end{corollary}
\begin{proof}
    $G$ contains some non-identity element since $|G|=p>1$.
    Choose any $g\in G$ such that $g\neq e$.
    Then $1<\operatorname{ord}g||G|=p$, so since $p$ is prime, then $\operatorname{ord}g=p$.
    Thus since $\langle g\rangle\le G$ and they are both finite and of the same order, $C_p\cong \langle g\rangle=G$.
\end{proof}
There is a corollary of this in number theory.
We consider the collection of all (equivalent classes) of integers $k\in\mathbb Z_n$ such that $(k,n)=1$.
Since $(k,n)=1$, $k$ is invertible in $\mathbb Z_n$ for any such $k$ in the collection.
Conversely, if $k$ cannot be invertible in $\mathbb Z_n$ if $(k,n)\neq 1$.\\
So this collection is a group $\mathbb Z_n^{\times}$ under multiplication modulo $n$.
Also, this group has order $\phi(n)$ which is the number of positive integers less than $n$ that are coprime to it.
\begin{corollary}[Fermat-Euler Theorem]
    If $(k,n)=1$, then
    $$k^{\phi(n)}\equiv 1\pmod{n}$$
\end{corollary}
\begin{proof}
    Apply the preceding corollary to $\mathbb Z_n^\times$.
\end{proof}
After these warm-ups, we shall prove the Orbit-Stabiliser Theorem (in a stronger form).
\begin{theorem}[Orbit-Stabiliser Theorem]
    Suppose a group $G$ acts on a set $X$ and $x\in X$, then there is a bijection $G/G_x\to G\star x$ by $\phi:gG_x\mapsto g\star x$.
\end{theorem}
\begin{proof}
    To see $\phi$ is well defined, observe that $\forall h\in G_x, g\star x=g\star (h\star x)=(gh)\star x$.
    It is obviously surjective by the definition of orbit.
    It is injective since if $\phi(gH)=\phi(g'H)$, then $(g^{-1}g')\star x=x\implies g^{-1}g'\in H\implies gH=g'H$.
\end{proof}
\begin{corollary}
    Theorem \ref{ost_finite}.
\end{corollary}
\begin{proof}
    Due to the existence of this bijection, if $G$ is finite, then $|G|/|G_x|=|G/G_x|=|G\star x|\implies |G\star x||G_x|=|G|$.
\end{proof}
One of the most important application of this theorem is to work out the order of some finite group.
\begin{example}
    We look at the rotational symmetries of a cube.
    Collect the symmetries as the group $G$ and consider its action on the eight vertices $X$.
    Now this action is transitive, obviously, so fixing any $x\in X$, $|G\star x|=8$.
    Also, any member of the stabiliser of $x$ must be rotations though the axis through $x$ and the centre of the centre of the group since it fixes that.
    There are three rotations of this form, so $|G|=|G_x||G\star x|=24$.
\end{example}
\begin{theorem}[Cauchy's Theorem]\label{cauchy}
    Let $G$ be a finite group and suppose $p$ is a prime dividing its order, then $G$ contains an element of order $p$.
\end{theorem}
\begin{proof}
    Consider a subset $X\subseteq G^p$ defined by $X=\{(g_1,g_2,\ldots, g_p)\in G^p:g_1g_2\cdots g_p=e\}$.
    Since $|G^p|=|G|^p$, and $|X|=|G|^{p-1}$.
    Let $H=C_p=\langle\xi\rangle$, consider the action of $H$ on $X$ by
    $$\xi\star (g_1,g_2,\ldots,g_p)=(g_2,g_3,\ldots,g_p,g_1)$$
    This is an action, indeed, if $g_1g_2\cdots g_p=e$, then $g_2g_3\cdots g_pg_1=g_1^{-1}eg_1=e$.
    For any element $x\in X$, by Theorem \ref{ost_finite}, $p=|H|=|H_x||H\star x|$.
    But $p$ is prime, so every orbit has to have either size $1$ or size $p$, also the ordrs of the orbits sum to $|X|=|G|^{p-1}$ which is divisible by $p$.
    So the number of size $1$ orbits must be divisible by $p$, thus at least $2$.
    But all such orbits must be in the form $(g,g,\ldots, p)$, but since there are $2$ of them, there is some $g\neq e$ such that this tuple is in $X$, thus $g^p=e$ and $g\neq e$, therefore $g$ has order $p$.
\end{proof}
In fact, we have shown that the number of elements of order $p$ is congruent to $p-1\pmod{p}$.
\begin{definition}
    Fix a group $G$.
    $a,b\in G$ are conjugates of each other if $\exists g\in G, a=gbg^{-1}$.\\
    The conjugation action is an action of a group $g$ on itself by $g\star h=ghg^{-1}$.
\end{definition}
One can check that the conjugation is indeed an action.
\begin{definition}
    The orbits of the conjugation are called the conjugacy classes of $G$.
    The stabiliser of the conjugation of an element $h$ is called the centraliser $C_G(h)$ of $h$.
    The kernel of the conjugation action is called the centre $Z(G)$ of $G$.
\end{definition}
We can extend the conjugation action to the subgroups of $G$.
\begin{definition}
    If $H\le G$, then the conjugate of $H$ by $g$ is the subgroup $\{ghg^{-1}:h\in H\}$.
\end{definition}
It is trivial that the conjugate of a subgroup is indeed a subgroup.
