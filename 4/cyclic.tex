\section{Cyclic Groups}
Recall that $C_n$ is the set of $n^{th}$ roots of unity.
If we write $\xi=e^{2\pi i/n}$, the group is actually generated by $\xi$, that is, every element is of the form $\xi^k$ for some $k$.
Note that $\xi^n=\xi^0=1$.
\begin{definition}
    A group $G$ is called cyclic if there is an $a\in G$ such that every element is of the form $a^k$ for some $k$.\\
    The element $a$ is called the generator of $G$.
\end{definition}
\begin{example}
    1. The integers under addition is cyclic with generator $1$.\\
    2. The group $\mathbb Z_n$ under addition modulo $n$ is cyclic with generator $1$.
    But in fact, if we take the function $\phi(k)\to\xi^k$, this is an isomorphism and hence $C_n\cong\mathbb Z_n$.
\end{example}
\begin{theorem}[Classification of Cyclic Groups]
    A cyclic group is isomorphic to either $C_n$ for some $n\to\mathbb N$ or $\mathbb Z$.
\end{theorem}
\begin{proof}
    Let $G$ be a cyclic group and $a$ be its generator.
    Consider $S=\{k\in\mathbb N\setminus\{0\}:a^k=e\}$.
    If $S\neq\varnothing$, then let $n$ be the smallest element of $S$.
    Consider the function $\phi:C_n\to G$ by $\phi(\xi^k)=a^k$.
    We want to show that its an isomorphism.\\
    Now if $k,l<n$ are such that $k+l<n$, then $\phi(\xi^k\xi^l)=a^{k+l}=a^ka^l=\phi(\xi^k)\phi(\xi^l)$.
    On the other hand, if $k+l=n+r,0\le r<n$, then $\phi(\xi^k\xi^l)=a^{k+l}=a^{n+r}=a^r=\phi(\xi^r)=\phi(\xi^{n+r})=\phi(\xi^k)\phi(\xi^l)$.
    As $G$ is generated by $a$ and $a^n=e$, every element of $G$ is of the form $a^k$ for some $0\le k<n$, so $\phi(\xi^k)=a^k$, So $\phi$ is injective, consider the kernel of $\phi$.
    Note that if $\phi(\xi^k)=e$ then $a^k=e\implies k=0$, so $\ker\phi=\{1\}$, hence it is injective.
    So $G\cong C_n$.\\
    Now if $S=\varnothing$, then we shall show that $G\cong\mathbb Z$.
    Consider the map $\phi(k)=a^k$, then $\phi(k+l)=a^ka^l=\phi(k)\phi(l)$.
    This is surjective by the same argument as above.
    Its kernel consists of integers $k$ with $a^k=e$ but since $S$ is empty, $k=0$, so it is injective.
    Therefore $G\cong\mathbb Z$.
\end{proof}
Because of this theorem, it is convenient to write $\mathbb Z=C_\infty$.
\begin{definition}
    Let $G$ be a group and $g\in\mathbb G$, then the order of $g$ is the smallest positive integer $n$ such that $g^n=e$ if it exists.
    If there isn't such an $n$, then we say that $g$ has infinite order.\\
    We write $\operatorname{ord}(g)$ to denote the order of $g$.
\end{definition}
Consider the set generated by the powers of $g$.
It follows easily that this set is a subgroup of $G$, we denote this by $\langle g\rangle$, the subgroup generated by $g$.
It is cyclic, so it is isomorphic to $C_n$ where $n=\operatorname{ord}g$.