\section{Definitions}
Note, in our previous two examples of rotational symmetrical groups, the notion of composition is quite important:
In both of our proofs that the tetrahedron group and the isocagonal cone group are different, we have harness this notion.
Naturally, we should have it in our definition of group.
\begin{definition}
    Let $X$ be a set. A binary operation $\cdot$ is a function $:X\times X\to X$.
\end{definition}
\begin{definition}
    A group $G$ is a triple $(G,\cdot,e)$, where $G$ is a nonempty set, $\cdot$ is a binary operation on $G$ and $e\in G$, that satisfies\\
    G1. (Law of Associativity) $\forall a,b,c\in G, (a\cdot b)\cdot c=a\cdot (b\cdot c)$.\\
    G2. (Identity) $\forall a\in G, a\cdot e=a$.\\
    G3. (Inverse) $\forall a\in G, \exists b\in G, a\cdot b=e$
\end{definition}
Most of the time we write $a\cdot b$ as $ab$.
\begin{theorem}
    Let $(G, \cdot, e)$ be a group, then\\
    1. $ab=e\implies ba=e$.\\
    2. $ea=a$.\\
    3. $ab=e\land ab^\prime=e\implies b=b^\prime$.\\
    4. $\exists a, ae^\prime=a\implies e^\prime=e$.
\end{theorem}
\begin{proof}
    1. Choose $c$ such that $(ba)c=e$, so $e=bac=beac=b(ab)ac=ba((ba)c)=bae=ba$.\\
    2. By part 1, we can choose $b$ such that $ab=ba=e$, so $ea=aba=ae=a$.\\
    3. $b=bab^\prime=eb^\prime$ (due to part 1) $=b^\prime$ (due to part 2).\\
    4. By part 1, we choose $b$ such that $ba=e$, so $e^\prime=ee^\prime$ (by part 2) $=bae^\prime=ba=e$.
\end{proof}
\begin{remark}
    There are a LOT of proofs to the preceding theorem.
    For example, the lecturer used a somewhat different proof for part 1 of the theorem.
    Therefore, you should try and come up with your own -- it's great fun.
\end{remark}
By part 3 and axiom G3, any $a$ in the group has a unique $b$ in the group such that $ab=ba=e$.
Then we write $a^{-1}=b$ for this element.
This is called the \textit{inverse} of $a$.
Note as well that $(a^{-1})^{-1}=a$.
In addition, $(ab)^{-1}=b^{-1}a^{-1}$.
\begin{definition}
    For any element $a\in G$, declare $a^0=e$.
    Say inductively that $a^n=a(a^{n-1})$ for a positive integer $n$.
    Similarly for a negative integer $n$, $a^n=(a^{-1})^{-n}$
\end{definition}
One can check that the usual laws of indices apply.
\begin{proposition}
    1. $a^na^m=a^{m+n}$.\\
    2. $(a^n)^m=a^{nm}$.
\end{proposition}
\begin{proof}
    Trivial.
\end{proof}
There are some `fake axioms' that we do not actually have to be stated.
For example, the definition of binary operation includes the axiom of closure.
However, we do have to verify the closure property to show that something is a group.